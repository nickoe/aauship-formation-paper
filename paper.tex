%\documentclass[a4paper,12pt]{IEEEtran}
\documentclass[a4paper,conference]{IEEEtran}
%\input{preamble}
\usepackage{natbib}
\usepackage{graphicx}
\usepackage[utf8]{inputenc} % So we can input Nicks name in the paper title!
\usepackage[T1]{fontenc}
\usepackage{amsmath}%,amsfonts,amssymb} % Added so we can do pretty math equations.
\usepackage{geometry}
\usepackage{lipsum}
\usepackage[pdftex,bookmarks=true,bookmarksnumbered=true]{hyperref}
\usepackage[english,final]{varioref} % Vi kan anvende \vref
\usepackage[draft]{fixme}
\usepackage[boxed,linesnumbered]{algorithm2e}
%\usepackage{glossaries}
%\setacronymstyle{long-sc-short}
\geometry{left=3cm,right=3cm,bottom=4cm}
\newcommand{\colwidth}{3in}
\begin{document}

%\newacronym{FRP}{FRP}{Formation Reference Point}
%\newacronym{GNC}{GNC}{Guidance, Navigation and Control}
%\newacronym{LOS}{LOS}{Line-Of-Sight}
\title{\vspace{-2cm}Formation Control of AAUSHIP}
\author{Nick Østergaard \and Jeppe Dam \and Jesper A. Larsen}
\maketitle

%\begin{center}
%\vspace{-0.7cm}
%Group 12gr730
%\end{center}
%\thispagestyle{empty}

\begin{abstract}
Many maritime mapping tasks are today carried out by large research ships, which are very costly to operate. As a way to overcome this, a number of small surveying vessels have been developed called AAUSHIP. In order to efficiently map the an area with such smaller vessels, it is important that several vessels are able to corporate on the task at hand. In this paper, the developed formation control strategy for the AAUSHIP series of vessels is presented, along with simulation results, which confirms, that the algorithm works as intended.
\end{abstract}
\vspace{-3mm}
\section{Introduction}
The background for the formation control subject of this project
originate in a collaboration with Port of Aalborg, who has a vision to
make their harbour an intelligent harbour. This will, among other
things, include autonomous piloting of cargo ships bringing cargo to and from
Aalborg. For the cargo ships to enter the harbour it is important that
the seabed is deep enough and the sand has not build up larger bars or
moved the channel unexpectedly.  Currently bathymetry surveys are
performed manually with a small manned survey boat equipped with a
multi beam sonar, scanning some area of interest, which is a smaller
fraction of the whole Limfjord.

This is done with a period between three months up to three years,
depending on how active the seabed is. If the level is too shallow,
such that the cargo ships cannot enter, it is the Port of Aalborg that
needs to clear the area and ensure safe travel for their customers.

The work within this project is carried out to assist the Port of
Aalborg with their survey task. The development and implementation of
the AAUSHIP project will fit very well into this environment and be of
good aid for the Port of Aalborg.

The first focus point of the project is to model and test the
prototype of the AAUSHIP and then extend the fleet with duplicates of
the first AAUSHIP. The ship needs to follow a trajectory and thereby
sail within a predetermined location of interest. The second focus
point is to implement formation control of a fleet of AAUSHIP's and
test this at the location of interest. An area of the harbour has been
given as a use case to test the results against.

\section{Method}
The AAUSHIP is modelled by a 5 degree of freedom (DOF) model, which differs
from a 3 DOF model by including the pitch and roll also.
These are taken into account due to the fact that the AAUSHIP runs
with single beam sonars and therefore it is important to know the
relative pitch and roll angles. 

\subsection{Simulation Model}
\label{ch:simulation-model}
In order to simulate the ships behaviour in water, an accurate
simulation model has been developed and verified agains the real ship.
The hydro-dynamic model used to simulate the ships is given as:
\begin{align}
M_A \dot \nu_r + C_A(\nu_r)\nu_r + D(\nu_r)\nu_r + g(\eta_r) = \tau
\label{eq:hydmodel}
\end{align}
%\nomenclature{$M_A$}{The added mass matrix}
where
$M_A$ is the added mass matrix from the system, $C_A$ is the added mass matrix due to the Coriolis force, $D(\nu)$ is a combination of the potential and viscous damping matrices, $g(\eta)$ is the restoring forces, which is dependent on the position of the vessel, $\tau$ is control and propulsion forces and $\nu$ is the velocities of the vessel in all directions and moments.

The rigid body is used to model the physics of the vessel, as it is
assumed that the vessel is sufficiently stiff to neglect bending
dynamics. Translational motion and rotational motion can be derived by
analysis of this, and by ~\citep{sname1950} and ~\citep[sec.
(3.3.1)]{fossen} written in component form as:
$f^b_b = [X,Y,Z]^T$ and $m^b_b = [K,M,N]^T$, force through $o_b$ and moments about $o_b$ expressed in $\{b\}$,
$v^b_{b/n} = [u,v,w]^T$ is the linear velocity of $o_b$ relative $o_n$ expressed in $\{b\}$,
$\omega^b_{b/n} = [p,q,r]^T$ is the angular velocity of ${b}$ relative to $\{n\}$ expressed in  $\{b\}$ and 
$r^b_g = [x_g,y_g,z_g]^T$ is the vector from $o_b$ to CG expressed in $\{b\}$

The rigid body forces are written as:
\begin{align}
M_{RB} \dot \nu_r + C_{RB}(\nu_r)\nu_r = \tau_{RB}
\label{eq:rigidmodel}
\end{align}
where $M_{RB}$ is the system inertia matrix, $C_{RB}$ is the coriolis-centriopedal matrix, $\tau_{RB}$ is a lumped force combined of $\tau_{\text{hyd}}$ + $\tau_{\text{hs}}$+ $\tau_{\text{wind}}$ + $\tau_{\text{wave}}$.

Combined, this gives the following full vessel model:
\begin{align}
&\underbrace{M_{RB} \dot \nu_r + C_{RB}(\nu_r)\nu_r}_{\text{rigid-body forces}}\\\nonumber
&+ \underbrace{M_A \dot \nu_r + C_A(\nu_r)\nu_r + D(\nu_r)\nu_r + g(\eta_r)}_{\text{hydrodynamic forces}}  = \tau + \tau_{RB}
\label{eq:totmodel}
\end{align}
Since the vessel within this project is of smaller scale, the $C_A$ and $C_{RB}$ from \ref{eq:hydmodel} and \ref{eq:rigidmodel} are neglected ~\citep[eq. (2.23)]{fullactuatship}.

%\section{Control model}
%\fixme{include something about the linearized model here}


\section{Formation control}
\label{sc:one-approach}
In the following approach a potential field is generated for each
agent including obstacles, formation span, desired, and actual
position.  It will be a combination of virtual leader and potential
field. The principle generates a potential field to keep the formation
and that field is moved around as a virtual leader. When the virtual
leader is moved around it results in a deflection of the desired
position and causes the affected agents to get back into position. The
positions of the agents in the field is given individually to the
specific agents relative to the virtual leader. The approach generates
a single resulting vector for each agent which is used to guide the
agent. The potential field for each agent is generated from four
components:
\begin{align}
\tilde{\mathbf{F}}_i^{tot} = \mathbf{F}_{vl}+\mathbf{F}_{ij}^{tot}+\mathbf{F}_{ca}^{tot}+\mathbf{F}_{oa}^{tot}
\end{align}
where:
\begin{itemize}
%\firmlist%
\item[$\mathbf{F}_{vl}$] virtual leader force
\item[$\mathbf{F}_{ij}^{tot}$] inter-agent forces
\item[$\mathbf{F}_{ca}^{tot}$] agent-agent collision avoidance forces
\item[$\mathbf{F}_{oa}^{tot}$] agent-obstacle collision avoidance forces
\end{itemize}

\subsubsection{Virtual Leader, $\mathbf{F}_{vl}$}
The virtual leader is an anchor of each formation, the Formation Reference Point (FRP), and
controls the movement of this. This movement can be given as a
full trajectory of as a set of way points. The local virtual leader's
contribution to the field is defined as:
\begin{align}
\mathbf{F}_{vl} &= K_{vl}(p_{vl}^n-p_i^n-[p_{vl}^n-p_{i0}^n])\\
&= K_{vl}(\mathbf{d}_i-\mathbf{d}_{i0})
\end{align}
$K_{vl}$ is a tuning parameter. $p_{vl}$ is position of the virtual
leader, $p_i$ is position of agent $i$, $p_{i0}$ is desired position
of agent $i$ and the $d$ is a shorter notation for the distances in
between. The virtual leader component guides the agents directly to
their desired positions relative to the virtual leader.

\subsubsection{Inter Vehicle Influence, $\mathbf{F}_{ij}$}
This is the contribution of other vehicles to the potential field,
which is expressed as:
\begin{align}
\mathbf{F}_{ij} &= K_{ij}(p_{j}^n-p_i^n-[p_{j0}^n-p_{i0}^n])\\
&= K_{ij}(\mathbf{d}_{ij}-\mathbf{d}_{ij0})
\end{align}
Similar to previously the $p$s are positions, $K_{ij}$ is a tuning
parameter and $d$ is a shorter notation for the distances in between.
This component preserves the formation by affecting the agents to keep
their respective desired distances among themselves. The weighting on
each goal can be adjusted by $K_{vl}$ and $K_{ij}$, hence this
weighting is a weighting that causes the agents to either follow the
virtual leader or to preserve their desired formation.  In a swarm of
$N$ agents the total field for agent $i$ given by:
\begin{align}
\mathbf{F}_{ij}^{tot} = \sum\limits_{j=1}^N\mathbf{F}_{ij}(i,j) \text{ for } j\neq i
\end{align}

\subsubsection{Collision Avoidance, $\mathbf{F}_{ca}$}
The collision avoidance takes effect when the agents get closer than a
pre defined distance of each other. It generates an additional field
component for the vehicle $i$ which points away from the entering
agent causing the agents to move away from each other. To ensure the
avoidance the component converges towards infinity in the centre of
the $i$'th agent. The $\mathbf{F}_{ca}$ is expressed as:
\begin{align}
    \mathbf{F}_{ca}^{ij}= 
\begin{cases}
		\left(
    \frac{K_{ca}r}{||\mathbf{d}_{ij}||}-K_{ca}
		\right)
		\frac{\mathbf{d}_{ij}}{||\mathbf{d}_{ij}||}
		,& \text{for } ||\mathbf{d}_{ij}||<r\\
    0,              & \text{otherwise}
\end{cases}
\end{align}
where $K_{ca}$ is a tuning parameter. $r$ is the safety radius for
collision and $\mathbf{d}_{ij}$ is the distance between the individual agents.
The collision avoidance can be expressed in a total term of the
collision avoidance:
\begin{align}
\mathbf{F}_{ca}^{tot} = \sum\limits_{j=1}^N\mathbf{F}_{ca}^{ij} \text{ for } i\neq j
\end{align}

\subsubsection{Obstacle Avoidance, $\mathbf{F}_{oa}$}
The same principle as for collision avoidance can be applied to
obstacle avoidance. Now each obstacle needs to be handled as an agent,
which will make the same result, but the reference is a little
different:
\begin{align}
    \mathbf{F}_{oa}^{ik}= 
\begin{cases}
    \left( \frac{K_{oa}}{||\mathbf{d}_{ki}||}-\frac{K_{oa}}{r}\right)
		\frac{\mathbf{d}_{{ki}}}{||\mathbf{d}_{ki}||},& \text{for } ||\mathbf{d}_{ki}||<r\\
    0,              & \text{otherwise}
\end{cases}
\end{align}
where $k$ denotes the counter for obstacles instead of other agents.
$K_{oa}$ is also a tuning parameter for the obstacle avoidance.
$\mathbf{d}_{ki}$ is the vector between an agent and the obstacle, which in a
total term is summed up as:
\begin{align}
\mathbf{F}_{oa}^{tot} = \sum\limits_{k=1}^M\mathbf{F}_{oa}^{ik} \text{ for } i\neq k
\end{align}
Here $d_{ki}$ represents one of the $M$ place vectors which has the
effect of a detected obstacle.

It can be noticed that there is a difference between $\mathbf{F}_{ca}$
and $\mathbf{F}_{oa}$. The two forces applies the same directions of
forces, making the agents repulse from another agent or an obstacle, but for equal parameters the $\mathbf{F}_{ca}$ generates a larger force, meaning that collision is punished harder than obstacles. 


%Though there is made a difference between them. It can be seen that
%the safety radius is moved from a multiplication to a division from
%the $\mathbf{F}_{ca}$ to the $\mathbf{F}_{oa}$. This will, dependent
%on the choices of $K_{ca}$ and $K_{oa}$, allow agents to move further
%into the safety radius of an obstacle than into the radius of another
%agent. This is due to the fact that the repulsing forces from agents
%needs to be larger in the case that two agents moves toward each
%other. In this case the agents need to react with higher aggression to
%ensure that they will repulse enough to not collide. This will not be
%the case for obstacles since these are defined with a constant
%position. The 2-D representation of $Fca$ and $Foa$ in their safety
%radius can be seen on figure \ref{fig:fcafoa}.
%\begin{figure}[htbp]
%  \includegraphics[width=\colwidth]{fig/fcafoa}
%  \caption{The magnitude of the forces $Fca$ and $Foa$. This force applies when an agent or an obstacle enters within the safety radius.}
%  \label{fig:fcafoa}
%\end{figure}

%It can be seen that the force of $Fca$ are greater than the force of
%$Foa$, although their gains $Kca$ and $Koa$ have been chosen equally.
%This is due to the difference of their structure and ensures that no
%agents collide even though they are on a collision course.

The distance $r$ can be determined
dynamically depending on the velocity of the agent:
\begin{align}
r = r^{min} + K_r||\dot{\mathbf{p}^n}||
\end{align}
Still will ensure that the agents have the possibility to decelerate
from their absolute velocity in the safety radius such that they can
turn away from each other.


\subsubsection{Potential field}
The forces are summed together to get $\tilde{\mathbf{F}}_i^{tot}$,
which is an intermediate vector which gives the magnitude and
direction of the potential field for vehicle $i$ at its current
position.
\begin{align}
  \mathbf{F}_i^{tot} = \min\{\,\,\,||\tilde{\mathbf{F}}_i^{tot}||\,\,\,,\,\,F_{max}\,\,\,\}\frac{\tilde{\mathbf{F}}_i^{tot}}{||\tilde{\mathbf{F}}_i^{tot}||}
\end{align}
The $\tilde{\mathbf{F}}_i^{tot}$ denotes that it is a middle variable
and not the final value of the potential calculation, thus not the one
used by the controller yet.  As the potential field does not need to
expand to infinity it is reasonable to define a maximum amplitude for
the vector, while still keeping its direction, $F_{max}$. This will be
a limitation of the agents' speed. As a start in the simulation phase
is $F_{max}$ chosen as a constant, but in the fully implemented system
it can be of benefit to adjust this maximum speed dynamically, for
instance as in equation \ref{eq:dynfmax}, as an example from
\citep{UAVff3dpf}.
\begin{align}
F_{max} = F_{min} + K_{vl}||\dot{p}^n||
\label{eq:dynfmax}
\end{align}
where the $F_{min}$ is a minimum value for the upper limit and then
with an applied gain of the speed.  The reference trajectory is used
by the controller to calculate the agent's control input which can be
based on the desired movement in the NED frame as:
\begin{align}
  \mathbf{p}_{i,r}^n = \mathbf{p}_i^n + \mathbf{F}_i ^\text{tot}
\end{align}
where their positions are added together with the potential field,
such that the position and the potential field becomes linked.

\section{The Potential Field Strategy}
 The potential field is
generated for each individual agent at every update step to make the
formation move and converge to a specified formation and position. The
field is generated based on forces acting in an overlying potential
field structure where one force converges the agent to a desired
position, a force attracting the agent to obtain the desired formation
along the trajectory, a force repelling the agent from other agents if
their distance is too small and finally a force repelling the agents
from static objects. The latter two can seem the same, but the
repelling force will be larger for the agent-agent force due to the
fact that two agents could have course directly toward each other and
a more aggressive avoidance can be needed.

To be able to generate and simulate the potential field the
implementation needs to be generic. First it was developed with one
agent that needs to converge to a desired position and afterwards were
other agents added as obstacles and some static objects were added in
extend. From these obstacles it can be seen that a single agent is
able to converge to a position which makes it possible to expand such
that more agents can converge into formation with reference from
either a virtual leader or from each other. This will solve the
formation coordination task, where the following task will be the
group coordination task. The group coordination task has the goal to
move the formation around, which here will be done by making the
virtual leader, or an actual leader of the formation, follow a
specified trajectory. This will make the other agents follow this
leader and keep their formation on the trajectory.

\begin{figure}[htbp]
  \includegraphics[width=.9\linewidth]{fig/ftotmagnfigpdf1}
	\caption{Plot of one agent's trajectory with a desired position with
		obstacles to avoid}
  \label{fig:potfieldagenti}
\end{figure}

A plot for a total potential reference field for a single agent can be
seen on figure \ref{fig:potfieldagenti}. The red line made of crosses
is the trajectory that one single agent will follow, if the obstacles
to avoid in the plot are static. In the plot every object, either
another agent or an object, are kept static. So it shows how the
trajectory will be in one single time step. This will change in the
next time step if the other agents also move in the potential field.
The agent avoids obstacles on the way, where it can be seen that it
does not get into the safety radius of the obstacles. In this specific
plot is a safety radius ($r$) of $20$m chosen, such that the distance
from agent $p_i$ (red trajectory) to any obstacle always will be
larger than $20$m.

%The gains $K_{ca}$ and $K_{oa}$ are chosen equally to a constant as
%240. If this gain is chosen smaller it will result in that the agents
%are more allowed to approach the obstacles and get a little within the
%radius, but then afterwards getting repelled from the object. It
%corresponds to the gradient at the radius and how steep this is.

The same algorithm is applied where agent $i$ avoids other agents,
agent $j$, $j+1$. This can be seen on figure \ref{fig:avoidagent}.

\begin{figure}[htbp]
  \includegraphics[width=.9\linewidth]{fig/ftotmagnfigpdf}
	\caption{Agent $i$ avoids agent $j$ and converges to the minima at
		the virtual leader}
  \label{fig:avoidagent}
\end{figure}

Agent $i$ takes a direct course toward the virtual leader but meets
another agent as an obstacle. Agent $i$ moves on the boarder of agent
$j$ with the defined safety radius and afterwards diverges from agent
$j$ towards the virtual leader. This is all done by following the
lowest gradient at all times.

The gain of $K_{ij}$ is not to be interpret from figure
\ref{fig:potfieldagenti}. $K_{ij}$ is the gain to the force that
attracts the agents together by minimizing the distance in between
them. By doing this the agents will get faster into the desired
formation. The gain $K_{vl}$ does at some point the opposite. This
gain adjusts the weighting of how fixated the agents should be to
converge to the desired position. If this gain is relatively larger
than $K_{ij}$ then the agents will converge directly to their position
around the virtual leader and not converge to the desired formation on
the way. This implies that the scaling between $K_{vl}$ and $K_{ij}$
controls if the formation should converge to the desired formation on
the way to the desired position, or if agent $i$ should only have the
desired position in focus.

\subsection{Numerical Solution}
\label{subsec:numsol}
The grid in which the potential field is generated are limited with a
certain resolution while simulating the agents movement. This reduces
the directions of where the agents can move, which will not arise a
problem on the same level when implemented in reality. In the
simulation environment it reduces the resolution such that a single
field in the grid contains one value of magnitude of the potential
field, which makes the basis of a certain gradient to the field. The
agents are following the implementation of the steepest decent. This
generates a gradient towards the steepest decent, which the agent
tracks. The analogy can be seen as a bowl, or sphere in this case,
where a ball will converge towards the lowest point in the direction
of the minimum gradient.

The method of applying the grid with magnitude of the potential field
rises a problem with resolution, and therefore also a problem that
makes the 'corners' of the grid around the agent to have the steepest
decent. This is seen as if the agent is placed in the middle of a
3-by-3 matrix, and have eight placements around it. The placements
around the agent will then be checked. The magnitude of the vector
from the agent and outgoing will therefore be biggest in the corners
since the distance to those are greater than the distances to the
sides, up and down. This problem has been expanded with a solution
such that a certain radius in the potential field around an agent will
be checked. The value at the radius around the agent can be checked,
and due to the newly equal distance to every point, these will be
weighted equally with respect to their value. This makes in principle
the possibility to make the agent go in all directions which will be
closer to the reality. When testing the two methods against each other
it is clear that the first proposed with the grid structure did not
have the same mobility thus not preferable in simulations though it is
simpler. The first made the agents move only in the diagonals of their
local placement, where the latter makes the agents able to move in a
number of directions specified in the algorithm.

\subsection{Local Minima Problem}
A problem that can become crucial arises when two agents or two
objects are within the radius of each other. This will result in a
local minima in the potential field between those objects. This will
create a local minima in between these agents or objects. If an
agent converges toward this minima they cannot get out again. The
problem can be seen on figure \ref{fig:roevproblem}. The gains here
are chosen exactly the same as in figure \ref{fig:potfieldagenti}.

\begin{figure}[htbp]
  \includegraphics[width=.9\linewidth]{fig/ftotmagnfigpdf3}
	\caption{An agent gets stuck due to a local minima between two other
	agents. The agent cannot get out of this minima unless the other two
	agents makes the space for the agent to pass through}
  \label{fig:roevproblem}
\end{figure}

The scenario on figure \ref{fig:roevproblem} has the following steps.
The agent $i$ moves in the direction of the steepest decent. Then it
gets to the border of another agent where it cannot go through thus
starts to go around this agent. The problem arises when agent $i$
reaches another agent on the way where it now has reached a local
minima. Now the steepest gradient will point at the position where the
agent already is thus making it think it has reached the end point.
Solutions to this problem can be formulated in different ways.

One solution could be to cluster the two objects together and instead
of making their potential field individually, then combine those
together and make an ellipsoid or even a circle formed obstacle of
those objects. This will ensure that the local minima disappears thus
not making an agent get stuck between those objects.

Another solution is to make an exception handler that can tell if
agent $i$ has reached the desired position. If it has not reached its
end point, and the position is constant on the same placement, it
perturbs the desired position of the agent until the direction of the
steepest decent changes more than a predefined value. This will mean
that the agent is out of the local minima and can continue on the
trajectory.  The solution of clustering the objects, that are too
close, can be seen on figure \ref{fig:solroevproblem}.

\begin{figure}[htbp]
  \includegraphics[width=.9\linewidth]{fig/ftotmagnfigpdf4}
	\caption{An agent that before was stuck now does not get into a
		local minima close to the agents, as it now sees the two other
		agents as one larger agent.}
  \label{fig:solroevproblem}
\end{figure}

Here the first solution is applied where the two agents, that were too
close to each other, have been clustered into one, seen from the
$i$'th agent. Now the local minima between the agents have been
neglected and the $i$'th agent can generate its trajectory around the
agents and continue to the endpoint of the potential field. The
algorithm checks if the distances between the agents are lower that $2
\cdot r$. If this is the case it means that the $i$'th agent cannot
generate a trajectory in between these agents, which can lead to a
local minima. Therefore is the agents that are too close combined into
one by generating the middle point between their positions and
generating a new radius. This makes a larger circle where the two
agents are in the subset. This circle will be larger depending on the
wanted safety radius thus rises the need to recheck the potential
field again after have generated a new combined agent. If the radius
of the new agent places it close to one of the single agents, these
also might need to cluster. Thus the algorithm needs to run until no
distances between agents are less than two times the radius.

The algorithm generating this combined agent can be seen in pseudo
code in algorithm \ref{al:clusteragents}.
\begin{algorithm}[h]
  \KwData{clustering of agents}
  initialization\;
  \If{$||\mathbf{p}_i , \mathbf{p}_j|| < 2 \cdot r$}{
  $\mathbf{p}_{j,new} \gets \text{mid point between}\ \mathbf{p}_i\ \text{and}\ \mathbf{p}_j$\\
  $r_{new} \gets \text{calc new r for}\ \mathbf{p}_{j,new}$\\
  delete $\mathbf{p}_i\ \text{and}\ \mathbf{p}_j \text{with}\ \mathbf{p}_{j,new}$
  }
	\caption{This pseudo code describes how agents that are too close to
		each other are getting clustered and seen as one. The algorithm
		can also be applied for obstacles in the potential
		field.\vspace{3mm}}
  \label{al:clusteragents}
\end{algorithm}

The algorithm works by first checking every distance between the agents to see if it is lower
than $2 \cdot r$. If the distance is lower, a new coordinate set needs
to be calculated. The coordinates for the $\mathbf{p}_{j,new}$ is
generated to the middle value of the two points
\begin{align}
\mathbf{p}_{j,new} = \frac{\mathbf{p}_i + \mathbf{p}_j}{2}
\end{align}
and afterwards the new radius for $\mathbf{p}_{j,new}$ can be found from
\begin{align}
r_{new} = \frac{||\mathbf{p}_i , \mathbf{p}_j||}{2} + r
\end{align}

In the end this results in that the every agent needs a magnitude and
a direction of which they should move. This will be given depending on
the total environment where the agents are manoeuvring, and will be
assigned by the gradient vector. When applying this formation strategy
a collision free movement is guaranteed which is one of the more
critical criteria to be fulfilled. 


\section{Results}
The potential field control system consists of multiple elements
as seen with the flow which is illustrated on the block diagram on
figure~\vref{fig:potentialfield_block}.
\begin{figure}[htbp]
\centering
\includegraphics[width=\colwidth]{fig/potentialfield_block}
\caption{Block diagram showing the iteration process of using the
potential fields for computation of the input vector}
\label{fig:potentialfield_block}
\end{figure}
The first block is the potential field generator. This is the
potential field calculation to compute the magnitude of the global
potential field. This information is passed to a trajectory generator,
which generates the reference trajectory. This reference trajectory is
where the $i$'th agent needs to move. This is passed to the controller
of the vessel, which then computes the input to the actuators on the
vessels, which in this case is done via a linear controller, which has been developed based on a linearized version of the vessel around its nominal operating point. The position of the vessels are then fed back, both to the
potential field calculation, the trajectory generation and the
controller. The potential field needs to be calculated from the
vessels relative position, the trajectory generation needs the
position for the intermediate reference position and the controller
will need it for i.e. error calculations.

Part of this flow can be computed by the $i$'th ship themselves, but
the overlying trajectory generation needs to be handled by the virtual
leader, or one leader in the formation.

The Guidance Navigation and Control (GNC) works by an array of mission specific waypoints given,
usually computed from a desired area used to create a lawnmower
pattern or similar area coverage algorihm. This trajectory
becomes the area of interest, and is the one overlying trajectory that
the virtual leader has to follow. The other ships will need to
maintain their individual positions at all time steps respective to
the virtual leader. Dependent of how the position is formed, the ships
needs to go into formation before or during the trajectory tracking
phase.

%\paragraph{Waypoint Database}
%Different methods can be used to steer ships after this
%path. The simplest is the usual heading autopilot, which will
%just steer the reading of the ship to the course angle to the
%waypoint. Or more elaborate, ways is the use of the waypoints as
%line segments that the ship should follow. This is implemented with a
%Line of Sight (LOS) algorithm. This algorithm, can work, but it is too simple to
%include obstacle or inter vehicle collisions. A way proposed to solve
%this issue is to use the concept of potential fields.%
%
%\paragraph{Potential Field}
%The potential field itself is merely some functions describing the
%repulsive and repelling forces between points of interest in the map.
%These points of interest are all objects that matters for the
%navigation, that is all ships, the anchor point (virtual leader) of
%the formation, and other point obstacles. This using the methodology
%described in the paper \citep{UAVff3dpf}.%

%The potential field is used in an iterative algorithm which can
%calculate the direction (from the $i$'th boat to the desired position
%spanned by a potential field defining the formation.

%\paragraph{Trajectory Generation}
%In the end, a reference path is calculated by the means of the
%previous position and the result from the potential field
%solver. It is calculated as the paper presents, \citep[eq.
%48]{UAVff3dpf} and described in section~\vref{sc:one-approach}.

%\begin{align}
%	\mathbf{p}_{i,r}^n = \mathbf{p}_i^n + \mathbf{F}_i ^\text{tot}
%\end{align}

%This is passed to the ships inner control loop.

%When the trajectory is generated from a series of way points, which is
%not ordered as a series of equidistant way points, it means that some
%handling of when or how to update the position of the virtual leader
%is needed.

%For example; in the case, where the way points are far from each
%other, it is still desired to make the virtual leader trajectory
%converge to the line between these two way points, kind of like the
%LOS guidance. Earlier a path following algorithm using a LOS
%principle was used, to calculate a course angle to a point projected
%%into the line between two way points with a specified lookahead%
%distance. The projection with the lookahead ensures convergence to the
%line. But in the case of controlling the virtual leader, it is not
%necessary to calculate the course angle from the virtual leader to the
%projected point, because this is only a virtual anchor of the
%formation, hence this only specifies where the geometry of the
%formation is calculated from. 

For the inner loop, a heading based LOS method can still be used,
but this should be calculated for every ship, with each their
reference position $\mathbf{p}_{i,r}$.

\begin{algorithm}[h]
	\KwData{track as global mission trajectory as way points}
	initialization\;
	\While{$m <=$ length of track}{
		\For{every $i$-th boat}{
			\If{formation is ok}{
				\If{$\mathbf{p}_{vl}$ is inside the way point acceptance radius of the
				track}{
					$m \gets m + 1$\; 
					$\mathbf{p}_{vl} \gets$ \textsl{LOS}$(\ p_{vl}\ ,\ $track($m$)\ )\;
				}
			}
			$(\ \mathbf{p}_{d,i}\ ,\ \mathbf{F}_{\text{tot},i}\ ) \gets $  \textsl{pathgen}(\ $p_i$\ ,\ $p0_i$\ )\;
			$\mathbf{p}_{r,i} \gets \mathbf{p}_i + \mathbf{F}_{\text{tot},i}$\;
			$\psi_{d,i} \gets $ heading from $\mathbf{p}_i$ to $\mathbf{p}_d$\;
			$u_i \gets $ \textsl{controller}(\ $\psi_d$\ )\;
			\textsl{send input $u$ to ship}\;
			$\mathbf{x}_i \gets $ \textsl{sense ship states}\;
			$\mathbf{p}_i \gets $ position of $\mathbf{x}_i$\; 
		}
	}
	\caption{This pseudo code describes how the potential field is used
	for each boat to calculate the reference for the inner controller
	for every boat at every time step. Every iteration in the while loop
	is a time step.\vspace{6pt}}
	\label{al:potfield}
\end{algorithm}

The algorithm~\vref{al:potfield} describes how the potential field
strategy can be simulated, were each iteration of the while loop is a
time step, which means that the control will continue until the
formation has reached the way point acceptance radius of the track.
This is to ensure that the formation anchor do not move forward if the
ships are not properly in formation. This is analogous with the group
coordination task as defined by \citep{thorvaldsen}.

A simulation of the algorithm with the dynamics of the AAUSHIP can be seen on figure \ref{fig:potform}.
\begin{figure}[htbp]
  \centering
  \includegraphics[width=\colwidth]{fig/lawn-bar-formation}
  \caption{Four agents are placed relative to the middle point of the
	formation, the virtual leader. This leader moves at the trajectory
with the waypoints, but only changes to the next waypoint if the
agents are in formation. The blue circles with the red plus signs are
the waypoints of the virtual leader. The other paths are computed from
this and the FRP.}
  \label{fig:potform}
\end{figure}
The red crosses are waypoints that have been targeted as the next waypoint to reach. The line connecting those waypoints are the virtual leader movement, which changes position from waypoint to waypoint to generate the straight line segments for the formation to follow. Every of the four agents have a relative position placement to the virtual leader, the agents positions are $p_{ij}$ and given as $p^n_{vl} + \text{offset}$ and the position of the virtual leader is given as $p^n_{vl}$. The ships are shown as yellow ships and the ships in formation is connected with a red line.

The formation have started at position $[-250,-150]$ and has the first waypoint in $[-208,-120]$. When the agents are close to reach the waypoint at $[-208,-120]$, they ensure that every agent are in formation by waiting for the last to catch up, if needed. Then all of them are in formation with respect to the virtual leader and this changes waypoint such that the agents needs to go toward the next waypoint. This waypoint shifting continues until no more waypoints are available.

It can be seen that there is a little divergence of the ships to the line segments which is mainly due to the dynamics of the AAUSHIP. Their respective line segments are not shown on the figure, while this would make the figure confusing. Though the agents will follow a line segment from their position in the formation to the new position in the formation. This is due to the movement of the virtual leader where this only moves in straight lines, thus the following agents pursue to do the same. The trajectory the ships follow is plotted beneath the third vessel from the left from where it can be seen that this vessel is placed on top of the virtual leader, and almost makes this vessel serve as the leader of the formation. Due to the formation setup the following ships will follow the same trajectory as the leader but only in this case shifted in the easting position.

\section{Discussion}

The model and simulation has proved to work when implemented on the AAUSHIP which results in a vessel that tracks a predetermined trajectory. This makes the basis for the further work of expanding the fleet of AAUSHIPS and implement the investigated control strategies and ending up with a successful digital mapping of the seabed in the Limfjord.

%For now the first aim is reached with a model of the AAUSHIP that can
%track the given trajectory, implementation of the whole ship is close
%to done, it just needs testing. Moving on the focus is to determine
%the control paradigms within formation control that fits with our
%use case and implement this on AAUSHIP and the rest of the fleet.

%As the complete
%conclusion should the tests be performed in the same location of
%interest as the data have been taken by the Port of Aalborg. The
%conclusion should by that time be that the fleet can do it as good as
%the manual scanning performed by The Port of Aalborg.

\bibliographystyle{apalike} 
\bibliography{bib}
\end{document}

%Here's a suggestion as to what an extended abstract should contain:
%
%    Background - A little history about who's done what and how your work fits in with it.
%    Aim - What you're trying to tell the audience that they don't already know (e.g. Your story.)
%    Method - Why the audience should believe that the results you've got aren't made up or flawed
%    Results - Evidence that you've come up with that confirms your story
%    Conclusion - Recap of your story and its implications
%    Limitations - Why someone might doubt your story and what you've done to get rid of as much doubt as possible.
%
%What if I'm presenting a review of my progress to date and I have no original research?
%Method = literature survey.
%Results = what you've read.
